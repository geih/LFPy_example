\documentclass[a4paper,12pt]{article}
\usepackage[utf8]{inputenc}
\usepackage{geometry}
\usepackage{fancyhdr}
\usepackage{amsmath}
\usepackage{wallpaper}
\usepackage{graphicx}
\pagestyle{fancy}
\fancyhf{}
\fancyhead[L]{LFPy-exercise BNNI 2019}

\begin{document}
\section*{Exercise 9:\\Forward modelling of extracellular potentials}

In this exercise you will work with forward modelling of extracellular potentials, based on the Hay model\footnote{Hay et al. (2011) {\it Models of neocortical layer 5b pyramidal cells capturing a wide range of dendritic and perisomatic active properties.} PLoS Comput Biol 7: e1002107, \texttt{doi:10.1371/journal.pcbi.1002107}}. You will be given a simple example script, \texttt{LFPy\_ex.py}, based on \texttt{LFPy} and \texttt{Neuron}, which you can adapt to solve this exercise. Figure 1 is made using the provided code.

%%%%
\begin{figure}[hb]
\centering
\includegraphics[width=12cm]{figure_127}
\caption{A somatic synapse induces a spike in the Hay model that is recorded by a virtual electrode 25 $\mu m$ away from soma. White noise with a Root-Mean-Square amplitude of 15 $\mu V$ has been added to the extracellular signal.}
\end{figure}
%%%%

\paragraph{(i)} Read through the code, and try to get a general overview of what it does. 

\paragraph{(ii)} Given a noise-level on the virtual extracellular electrode corresponding to white noise with a Root-Mean-Square of 15 $\mu V$, roughly how far away from soma will a spike from the Hay model be visible on an electrode?

\paragraph{(iii)} The local field potential (LFP) is typically assumed to be caused by synaptic input to large populations of cells. We will now look at the origin of the LFP. 

\subparagraph {(a)} Decrease the synaptic weight so that the cell does \emph{not} fire an action potential, and plot the extracellular amplitude on a dense 2D grid around the cell at the time corresponding to the maximum deflection in the somatic membrane potential. A plotting function for this is already implemented in  \texttt{LFPy\_ex.py}. 

\subparagraph {(b)} How does the shape of the LFP-amplitude grid depend on synapse position? Compute and plot the grid for various synapse positions, including 1) a somatic synapse ($y=0\ \mu m$ as in (a),  2) a synapse positioned at the apical dendrite ($y=600\ \mu m$), and 3) a synapse positioned at the distal apical dendrite ($y=1200\ \mu m$)? 

\subparagraph {(c)} Can you based on only the shape of the LFP-amplitude grids predict which of the three cases in b) that will fall off most rapid with distance from the cell?\\
{\it Hint: Distance decay of a dipole versus higher order multipoles.}



\end{document}
